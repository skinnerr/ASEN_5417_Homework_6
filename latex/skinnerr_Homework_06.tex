\documentclass[11pt]{article}

\input{../../Latex_Common/skinnerr_latex_preamble_asen5417.tex}

%%
%% DOCUMENT START
%%

\begin{document}

\pagestyle{fancyplain}
\lhead{}
\chead{}
\rhead{}
\lfoot{\hrule ASEN 5417: Homework 6}
\cfoot{\hrule \thepage}
\rfoot{\hrule Ryan Skinner}

\noindent
{\Large Homework 6}
\hfill
{\large Ryan Skinner}
\\[0.5ex]
{\large ASEN 5417: Numerical Methods}
\hfill
{\large Due 2015/11/12}\\
\hrule
\vspace{6pt}

%%%%%%%%%%%%%%%%%%%%%%%%%%%%%%%%%%%%%%%%%%%%%%%%%
%%%%%%%%%%%%%%%%%%%%%%%%%%%%%%%%%%%%%%%%%%%%%%%%%
\section{Introduction} %%%%%%%%%%%%%%%%%%%%%%%%%%
%%%%%%%%%%%%%%%%%%%%%%%%%%%%%%%%%%%%%%%%%%%%%%%%%
%%%%%%%%%%%%%%%%%%%%%%%%%%%%%%%%%%%%%%%%%%%%%%%%%

\begin{wrapfigure}[17]{R}{0.5\textwidth}
\vspace{-0.8cm}
\begin{center}
\begin{tikzpicture}
    \begin{axis}[
    width=0.5\textwidth,
    axis lines=middle,
    axis line style=thick,
    unit vector ratio=1 1 1,
    xlabel style={at=(current axis.right of origin), anchor=north east},
    ylabel style={at=(current axis.above origin), anchor=north east},
    domain=0:1.2,
    xmin=-0.05,
    xmax=1.2,
    ymin=-0.05,
    ymax=1.2,
    xlabel=$Z$,
    ylabel=$Y$,
    xtick={0,1},
    ytick={0,1},
    clip=false
    ]
    \coordinate (O) at (axis cs:0,0);
    \coordinate (A) at (axis cs:1,0);
    \coordinate (B) at (axis cs:1,1);
    \coordinate (C) at (axis cs:0,1);
    \draw[ultra thick] (O) -- (A) node[black,midway,below]{$U=0$}
                           -- (B) node[black,midway,right]{$U=0$}
                           -- (C) node[black,midway,above]{$\begin{aligned} U_{,Y} &= 0, &\text{Problem 1} \\
                                                                            U      &= 1, &\text{Problem 2} \end{aligned}$}
                           -- (O) node[black,midway,left]{$U=0$};
    \end{axis}
\end{tikzpicture}
\\[6pt]
\caption{Boundary value problem for Homework 6. No-slip conditions are imposed at three side walls. For Problem 2, the upper boundary is a moving wall. }
\label{fig:bvp}
\end{center}
\end{wrapfigure}

The equations that govern open channel flow at an inclination of $\theta$ degrees with the horizontal can be transformed into a Poisson equation by scaling the streamwise velocity $u$ as
\begin{equation}
U(Y,Z) = \frac{u(Y,Z)}{L^2 \rho g \sin(\theta / \mu)}
\;,
\end{equation}
where $L$ is the length of the square channel, $\rho$ is the fluid density, $g$ is gravitational acceleration, and $\mu$ is the fluid's dynamic viscosity. The cross-sectional dimensions in $y$ and $z$ are also normalized by $Y = y/L$ and $Z = Z/L$. Through these scaling procedures, the governing equations map onto a unit square as
\begin{align}
U_{,YY} + U_{,ZZ} &= -1
\;, \label{eq:poisson} \\
U(0,Z) &= 0
\;, \notag\\
U_{,Y}(1,Z) &= 0
\;, \notag\\
U(Y,0) = U(Y,1) &= 0
\;.
\label{eq:original_bcs}
\end{align}

\subsection{Problem 1}

Numerically integrate \eqref{eq:poisson} with the stated boundary conditions \eqref{eq:original_bcs} using the ADI method with $N = M = 101$ grid points in each direction. Implement LU decomposition to solve the tridiagonal systems, and determine convergence by a reduction of the original error by three orders of magnitude.

Plot the contours of $U$ in the $Y$-$Z$ plane at convergence.

\subsection{Problem 2}

Change the upper boundary condition to represent a solid lid moving at a constant velocity with
\begin{equation}
U_{,Y}(1,Z) = 0 \quad\longrightarrow\quad U(1,Z) = 1
\;.
\label{eq:new_bcs}
\end{equation}
Solve this problem using the successive over-relaxation (SOR) method. As in Problem 1, define convergence as a reduction by three orders of magnitude of the initial error in the maximum norm.

Obtain the best estimate for the acceleration parameter $\omega$ by numerical experimentation. That is, plot the number of iterations required for convergence as a function of $\omega$, and determine the value of $\omega$ that minimizes this function. How does this value of $\omega$ compare to the theoretical value?

Plot the contours of $U$ in the $Y$-$Z$ plane at convergence, and compare the results to Problem 1.

%%%%%%%%%%%%%%%%%%%%%%%%%%%%%%%%%%%%%%%%%%%%%%%%%
%%%%%%%%%%%%%%%%%%%%%%%%%%%%%%%%%%%%%%%%%%%%%%%%%
\section{Methodology} %%%%%%%%%%%%%%%%%%%%%%%%%%%
%%%%%%%%%%%%%%%%%%%%%%%%%%%%%%%%%%%%%%%%%%%%%%%%%
%%%%%%%%%%%%%%%%%%%%%%%%%%%%%%%%%%%%%%%%%%%%%%%%%

Before discussing the numerical methods, we note that the physical application of the source term $\xi$ (both problems) and the upper boundary condition $U=1$ (Problem 2) is somewhat confusing, due to the fact that the $Z$-$Y$ plane as defined is not right-handed. That is, $Z \times Y = -X$, rather than $+X$. As such, we opt to solve the differential equations as presented in \eqref{eq:poisson}, \eqref{eq:original_bcs}, and \eqref{eq:new_bcs}, rather than making physical arguments to correct for the coordinate system. As such, our solutions to the BVP may be off by a negative sign compared to their intuitive physical interpretations.

\subsection{Problem 1}

The alternating-direction implicit (ADI) method assumes a pseudo-time derivative ($\partial/\partial t$), such that \eqref{eq:poisson} becomes
\begin{equation}
T_{,t} = U_{,YY} + U_{,ZZ} - \xi = 0
\;,
\label{eq:adi_poisson}
\end{equation}
where $\xi$ is the inhomogeneous source term. Our solution for \eqref{eq:poisson} can be interpreted as the steady-state solution to \eqref{eq:adi_poisson} as $t \rightarrow \infty$. This equation is parabolic in space and elliptic in time. Of course, the transient is not physical, so it acceptable to advance the solution in time using the fully implicit Euler method.

The ADI method breaks the problem into two directions and solves each over two half-time steps. Discretizing the problem using second-order central differences, defining the inhomogeneous source term as $\xi(Y,Z) = -1$, and letting the grid spacing in both directions be equal ($h \equiv \Delta y = \Delta z$), we obtain
\begin{align}
U_{i+1,j}^{n+1/2} - (2 + \rho) U_{i,j}^{n+1/2} + U_{i-1,j}^{n+1/2}
&=
-U_{i,j+1}^{n} + (2 - \rho) U_{i,j}^{n} - U_{i,j-1}^{n} - h^2 \xi_{i,j}
\;, \\
U_{i,j+1}^{n+1} - (2 + \rho) U_{i,j}^{n+1} + U_{i,j-1}^{n+1}
&=
-U_{i+1,j}^{n+1/2} + (2 - \rho) U_{i,j}^{n+1/2} - U_{i-1,j}^{n+1/2} - h^2 \xi_{i,j}
\;,
\end{align}
where $\rho = h^2 / \Delta t$, and subscript commas indicate a separation of spatial indices rather than differentiation. The first pass loops over all $j$ values, and for each $j$ solves for $U$ at all $i$-locations. The second equation does the opposite. Initial guesses of the solution along the domain must be provided, but the RHS is always known and solution proceeds in the standard manner for implicit central differences. Boundary conditions are incorporated in the standard fashion as well, by modifying the RHS and diagonal terms as needed. To determine convergence, we define our error norm for the $n$\th iteration as
\begin{equation}
\epsilon_n = \frac{1}{\epsilon_1} \sum_{i=1}^N \sum_{j=1}^N \norm{U_{i,j}^{n} - U_{i,j}^{n-1}}
\;, \quad
n = 2, 3, ...
\;,
\end{equation}
and cease solution when $\epsilon_n \le 10^{-3}$. Here, $\epsilon_1$ is defined as the RHS sums evaluated for $n=1$, and the initial guess is defined as $U_{i,j}^0 = 0$.

Peaceman and Rachford (1955) show that the ADI method converges for any value of the iteration parameter $\rho$. The optimal value is a function of the iteration number $k$:
\begin{equation}
\rho_k = 4 \sin^2 \frac{k \pi}{2 N}
\;, \qquad
k = 1, 2, ..., n \quad \text{(until convergence)}
\;.
\end{equation}
Note that since we change $\rho$ on each iteration, the $\mb{L}$ and $\mb{U}$ matrices involved in LU-decomposition need to be re-assembled with each iteration, and we sacrifice some of the efficiency gains of LU-decomposition. The details of LU-decomposition will not be discussed here, as they were presented in full in Homework~4.

\subsection{Problem 2}

For the SOR method, we perform ``column sweeps'' over the whole domain until convergence is achieved. That is, we loop over $i = 2, ..., N-1$ and apply the following SOR central difference equation to each grid point with $j = 2, ..., M-1$:
\begin{equation}
U_{i,j}^{n}
=
\left( 1 - \omega \right) U_{i,j}^{n-1}
+ \frac{\omega}{4}
\left(
  U_{i-1,j}^{n}
+ U_{i,j-1}^{n}
+ U_{i+1,j}^{n-1}
+ U_{i,j+1}^{n-1}
- h^2 \xi_{i,j}
\right)
\;,
\end{equation}
where $\omega$ is the relaxation parameter, which typically has an optimal value in the range $[1.7,1.9]$.

\begin{wrapfigure}[15]{R}{0.5\textwidth}
\begin{center}
\includegraphics[scale=0.6]{Prob1.eps}
\\[0.5cm]
\caption{Contours of $U$ for Problem 1.}
\label{fig:Prob1}
\end{center}
\end{wrapfigure}

The theoretical optimum value of $\omega$ is given by
\begin{align}
\omega_\text{opt} &= \frac{8 - 4 \sqrt{4 - \alpha^2}}{\alpha^2}
\;,\\
\alpha &= \cos(\pi/M) + \cos(\pi/N)
\;,
\end{align}
and for our grid with $M=N=101$, we obtain $\alpha = 1.9990$ and $\omega_\text{opt} = 1.9397$.

We define our convergence criterion exactly the same as in Problem 1, though we could have used the maximum norm, which is defined as
\begin{equation}
\epsilon_n = \frac{1}{\epsilon_1} \max_{i,j} \norm{U_{i,j}^{n} - U_{i,j}^{n-1}}
\;,
\end{equation}
where $\epsilon_1$ is defined as the RHS max operation evaluated for $n=1$. The initial guess is defined as $U_{i,j}^0 = 0$ except for when $j=N$, in which case the upper boundary condition $U=1$ is imposed.

%%%%%%%%%%%%%%%%%%%%%%%%%%%%%%%%%%%%%%%%%%%%%%%%%
%%%%%%%%%%%%%%%%%%%%%%%%%%%%%%%%%%%%%%%%%%%%%%%%%
\section{Results} %%%%%%%%%%%%%%%%%%%%%%%%%%%%%%%
%%%%%%%%%%%%%%%%%%%%%%%%%%%%%%%%%%%%%%%%%%%%%%%%%
%%%%%%%%%%%%%%%%%%%%%%%%%%%%%%%%%%%%%%%%%%%%%%%%%

\subsection{Problem 1}

A contour plot of the solution for $U$ obtained from the ADI method is presented in \figref{fig:Prob1}.

\subsection{Problem 2}

Iterations to convergence as a function of $\omega$, as well as a contour plot of the solution for $U$ obtained from the SOR method is presented in \figref{fig:Prob2}.

Experimentally, we find $\omega_\text{opt} = 1.9385 \pm 0.0015$, which is within error of our theoretical $\omega_\text{opt} = 1.9397$.

\begin{figure}[h!]
\begin{center}
\includegraphics[scale=0.6]{Prob2_convergence.eps}
\includegraphics[scale=0.6]{Prob2_contours.eps}
\\[0.5cm]
\caption{Convergence behavior and contours of $U$ for Problem 2. Neighboring values of $\omega = 1.937$ and $\omega = 1.940$ both converge in 202 iterations.}
\label{fig:Prob2}
\end{center}
\end{figure}

%%%%%%%%%%%%%%%%%%%%%%%%%%%%%%%%%%%%%%%%%%%%%%%%%
%%%%%%%%%%%%%%%%%%%%%%%%%%%%%%%%%%%%%%%%%%%%%%%%%
\section{Discussion} %%%%%%%%%%%%%%%%%%%%%%%%%%%%
%%%%%%%%%%%%%%%%%%%%%%%%%%%%%%%%%%%%%%%%%%%%%%%%%
%%%%%%%%%%%%%%%%%%%%%%%%%%%%%%%%%%%%%%%%%%%%%%%%%

\subsection{Problem 1}

The ADI method converges very rapidly (within 5 iterations) using the variable time-step $\sigma_k$. The contours obtained are reasonable for laminar flow in a duct of rectangular cross-section with an aspect ratio of 2. In \figref{fig:Prob1}, we only see the lower half of this duct.

\subsection{Problem 2}

Substantial improvements are observed by using the SOR method compared to a simple Gauss-Seidel method, which the SOR method reduces to when $\omega = 1$. This value of $\omega$ is not shown in \figref{fig:Prob2} due to a high number of iterations (4142) to convergence. The experimental value of $\omega_\text{opt}$ matches the theoretical value within an error of 0.08\%. Physically, we observe the moving wall boundary condition $U=1$ to dominate the source term $\xi$ in dictating flow behavior.

%%%%%%%%%%%%%%%%%%%%%%%%%%%%%%%%%%%%%%%%%%%%%%%%%
%%%%%%%%%%%%%%%%%%%%%%%%%%%%%%%%%%%%%%%%%%%%%%%%%
\section{References} %%%%%%%%%%%%%%%%%%%%%%%%%%%%
%%%%%%%%%%%%%%%%%%%%%%%%%%%%%%%%%%%%%%%%%%%%%%%%%
%%%%%%%%%%%%%%%%%%%%%%%%%%%%%%%%%%%%%%%%%%%%%%%%%

No external references were used other than the course notes for this assignment.

%%%%%%%%%%%%%%%%%%%%%%%%%%%%%%%%%%%%%%%%%%%%%%%%%
%%%%%%%%%%%%%%%%%%%%%%%%%%%%%%%%%%%%%%%%%%%%%%%%%
\section*{Appendix: MATLAB Code} %%%%%%%%%%%%%%%%
%%%%%%%%%%%%%%%%%%%%%%%%%%%%%%%%%%%%%%%%%%%%%%%%%
%%%%%%%%%%%%%%%%%%%%%%%%%%%%%%%%%%%%%%%%%%%%%%%%%

The following code listings generate all figures presented in this homework assignment. The author's own LU-decomposition code is omitted since it was already presented in Homework~4.

\includecode{Problem_1.m}
\includecode{Assemble_ADI.m}
\includecode{Problem_2.m}

%%
%% DOCUMENT END
%%
\end{document}
